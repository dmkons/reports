The assignment was completed and the requirements were met.
Two additional design goals were set and prioritized by importance, which heavily influenced the solution.
The solution processor is capable of executing a comfortably large number of instructions in a performant manner, all the while following the recommended best practices put forth by IEEE and prominent VHDL gurus, as documented in \cite{assignment-1}.
The processor was successfully programmed onto the FPGA, but due to what was probably a fluke in the communication setup, data could not be read back from the running processor.

The assignment was suprisingly enough easier than than the previous assignment\cite{assignment-1}.
One reason for that is that the authors have become more procifient in VHDL and processor design in general, which has greatly helped speed up the design process.
A lot of the rookie mistakes that were made in \cite{assignment-1} were avoided in this assignment.
Additionally, for this assignment, a smaller implementation scope was wisely selected.

In conclusion, it was a fun assignment which was completed to reasonable satisfaction despite less than optimal time management within the group.
