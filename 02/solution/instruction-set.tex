The solution processor implements a modified and expanded subset of the MIPS instruction set.
A quick reference of the MIPS instruction set can be found in Figure 3.4 of the compendium~\cite[p.112]{compendium}.
The instructions, as in regular MIPS, can be on one of three general formats: R, I and J~\cite[p.109-11]{compendium}.

The solution processor implements the instructions in table \vref{table:implemented-instructions}.
Although implementing a large instruction set was not a design goal in this assignment, as it was in assignment \#1\cn, the processor still supports quite a few more instructions than the minimum requirements.
This is because of the addition of an assembler, which allows for pseudo-instructions.
The number of supported instructions in the instruction set all but doubles because of this.

\begin{table}[H]
    \begin{center}
        \begin{tabular}{r|l}
            \texttt{ADD} & Add \\
            \texttt{ADDI} & Add immediate \\
            \texttt{AND} & And \\
            \texttt{BNE} & Branch not equal \\
            \texttt{CMP} & Compare (pseudo) \\
            \texttt{JMP} & Jump \\
            \texttt{LDI} & Load lower immediate (pseudo) \\
            \texttt{LD} & Load word \\
            \texttt{MV} & Move (pseudo) \\
            \texttt{NEG} & Negation (pseudo) \\
            \texttt{NOP} & No operation (pseudo) \\
            \texttt{PASSTHROUGH} & Passthrough (i.e. send the X input unmodified through the ALU) \\
            \texttt{SUB} & Subtract \\
        \end{tabular}
        \smallskip
        \hrule
        \smallskip
        \caption{Instructions implemented in the solution processor}
        \label{table:implemented-instructions}
    \end{center}
\end{table}
