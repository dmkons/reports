The processor presented as a solution for assignment \#2 in this report is a simple 32-bit MIPS-inspired pipelined processor described in VHDL.
It was programmed onto a Xilinx Spartan-6 LX16 FPGA.

\section{Solution Architecture}

\todo{Generally describe the solution architecture}

\todo{Solution architecture image}

The rest of this section describes the different architectural subcomponents in detail.

\todo{detailed description of the components}

\subsection{ALU}

The ALU, or the arithmetic logic unit, is the heart of the processor.
The ALU is responsible for doing actual arithmetic and logical operations on data.
The rest of the processor is in reality machinery working to feed the ALU with as much data as quickly as possible.
It is therefore important that the ALU can work as fast as possible, in accordance to the performance design goal from section\vref{subsection:performance}.
The processor uses dedicated DSP slices on the FPGA to speed up the performance of the ALU.

\subsubsection{In Signals}

\begin{description}
\item{\textbf{Y}} \\
The first operand of an ALU operation.

\item{\textbf{X}} \\
The second operand of an ALU operation.

\item{\textbf{Function}} \\
The function code that decides which operation the ALU should perform on $ X $ and $ Y $.
\end{description}

\subsubsection{Out Signals}

\begin{description}
\item{\textbf{Result}} \\
    The result of the ALU operation.

\item{\textbf{Zero}} \\
    A Boolean value which is set if the result of the ALU operation is 0, or unset if the result of the ALU operation is not zero.
    This signal is used for determining the outcome of comparisons, so that they may be used in conditionals.
\end{description}

\newpage


\newpage
\section{Instruction Set}
\label{section:instruction-set}

The solution processor implements a modified subset of the MIPS instruction set.
A quick reference of the MIPS instruction set can be found in Figure 3.4 of the compendium \cite{compendium}.
The instructions, as in regular MIPS, can be on one of three general formats, R, I and J.

The solution processor implements the instructions in table \vref{table:implemented-instructions}.
The processor supports quite a few more instructions than the minimum requirements.
This is done in alignment with the design goals set for solution processor.

\begin{table}[h]
    \begin{center}
        \begin{tabular}{r|l}
            \texttt{ADD} & Add \\
            \texttt{ADDI} & Add immediate \\
            \texttt{ADDIU} & Add immediate unsigned \\
            \texttt{ADDU} & Add unsigned \\
            \texttt{AND} & And \\
            \texttt{ANDI} & And immediate \\
            \texttt{BEQ} & Branch if equal \\
            \texttt{BNE} & Branch not equal \\
            \texttt{J} & Jump \\
            \texttt{LUI} & Load upper immediate \\
            \texttt{LW} & Load word \\
            \texttt{MULT} & Multiply \\
            \texttt{MULTU} & Multiply unsigned \\
            \texttt{NOR} & Nor \\
            \texttt{OR} & Or \\
            \texttt{ORI} & Or immediate \\
            \texttt{PASSTHROUGH} & Passthrough (i.e. send the first input through unmodified) \\
            \texttt{SLL} & Shift left logical \\
            \texttt{SLLV} & Shift left logical variable \\
            \texttt{SLT} & Set less than \\
            \texttt{SLTI} & Set less than immediate \\
            \texttt{SLTIU} & Set less than immediate unsigned \\
            \texttt{SLTU} & Set less than unsigned \\
            \texttt{SRA} & Shift right arithmetic \\
            \texttt{SRAV} & Shift right arithmetic variable \\
            \texttt{SRL} & Shift right logical \\
            \texttt{SRLV} & Shift right logical variable \\
            \texttt{SUB} & Subtract \\
            \texttt{SUBU} & Subtract unsigned \\
            \texttt{SW} & Store word \\
            \texttt{XOR} & Xor \\
            \texttt{XORI} & Xor immediate \\
        \end{tabular}
        \smallskip
        \hrule
        \smallskip
        \caption{Implemented instructions}
        \label{table:implemented-instructions}
    \end{center}
\end{table}
 \label{sec:instruction-set}

\section{Assembler}

Bundled with the processor comes an assembler written in Python.
The modular and extensible assembler assembles the instruction set described in section \vref{section:instruction-set}.
It also supports labels and multi-line comments.
The assembler is hazard-aware, and will resolve use-after-load hazards by inserting nop instructions in the correct places to ensure correct execution on the processor.

Having an assembler also opens for the possibility of introducing pseudo-instructions, which are instructions that are converted into other instructions by the compiler.
As mentioned in section \vref{section:instruction-set}, a handful of pseudo-instructions are supported.

\section{Test Utilities}

The test utilities introduced as part of the solution to assignment \#1\cite{assignment-1} were ported and used in this assignment as well.
In particular, the tools in the VHDL package \texttt{test\_utils} were of great usefulness.


\section{Deliverables}

Here is the structure of the delivery:
\newline


\dirtree{%
.1 /.
.2 kons10/.
.3 report.pdf.
.3 code/.
.4 code.xise.
.4 driver/.
.5 host.py.
.5 main.c.
.5 user\_logic.vhd.
.4 hardware/.
.5 alu.vhd.
.5 branch\_controller.vhd.
.5 com.vhd.
.5 control\_unit.vhd.
.5 memory.vhd.
.5 mips\_constant\_pkg.vhd.
.5 mux.vhd.
.5 opcodes.vhd.
.5 PC.vhd.
.5 processor.vhd.
.5 register\_file.vhd.
.5 tb\_alu.vhd.
.5 tb\_branch\_controller.vhd.
.5 tb\_control\_unit.vhd.
.5 tb\_mux.vhd.
.5 tb\_pc.vhd.
.5 tb\_processor.vhd.
.5 tb\_toplevel.vhd.
.5 test\_utils.vhd.
.5 toplevel.vhd.
.4 xst/.
.5 proc\_common\_v3\_00\_a.
.5 proc\_common\_v3\_00\_a.ref.
}

The \texttt{code} folder contains the project code, while \texttt{report.pdf} is the report you are reading right this instant.	
