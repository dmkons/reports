\subsubsection{Pipeline Registers}

Between each stage of the pipeline there are series of registers that hold the necessary values between clock ticks.
The pipeline registers are clocked synchronously, which means that the pipeline design is a buffered synchronous design.
The pipeline registers are implemented as flip-flops.


\subsubsection{Flip Flop}

A flip-flop is a simple circuit that has two stable states.
The implication is that they can be used to store values over time.

In the solution processor, size-generic flip-flops VHDL are used.
Because type generics were first introduced in the VHDL 2008 standard\cite{vhdl2008}, and the Xilinx tool chain doesn't support it yet, flip-flops are implemented on a per-type basis.
Subsequently, the \texttt{flip\_flop.vhd} VHDL source code file is a lot bigger and hairier than it could be.

\subsubsection{Multiplexers}

A multiplexer is a component which selects between multiple input signals based on a separate control signal.
Since multiplexers are use quite commonly, and VHDL has no build in construct for it, some multiplexer components are introduced in the solution.

\subsubsubsection{Multiplexer 2}

The Multiplexer 2 selects one of two possible input values based on the value of a 1-bit selection signal.

\subsubsubsection{Multiplexer 3}

The Multiplexer 3 selects one of three possible input values based on the value of a 2-bit selection signal.
Since a 2-bit selection signal may have 4 possible states, and the Multiplexer 3 only has three inputs, one of the selection signal states ("\texttt{11}") holds undefined behaviour.

\subsubsection{Pipeline flusher}

\todo{write about this}
