The control unit is the component that is responsible for enabling and disabling the correct parts of the processor at the correct times, so that an instruction is executed correctly.

\subsubsection{In Signals}

\begin{description}
\item{\textbf{Instruction Op-code}} \\
    The op-code of the currently executing instruction in the instruction decode stage.

\item{\textbf{Instruction Function}} \\
    The ALU function of the currently executing instruction in the instruction decode stage. 
\end{description}

\subsubsection{Out Signals}

\begin{description}
\item{\textbf{Execute Control Signals}} \\
    The control signals that should be used in the execute stage for the instruction being decoded by the control unit.
    The execute control signals bus contains the \textbf{ALU Source}, \textbf{ALU Function} and \textbf{Register Destination} control signals.

\item{\textbf{Memory Control Signals}} \\
    The control signals that should be used in the memory stage for the instruction being decoded by the control unit.
    The memory control signals bus contains the \textbf{Branch}, \textbf{Jump} and \textbf{Memory Write} control signals.

\item{\textbf{Write-Back Control Signals}} \\
    The control signals that should be used in the write-back stage for the instruction being decoded by the control unit.
The write-back control signals bus contains the \textbf{Memory to Register} and \textbf{Register Write} control signals.
\end{description}

