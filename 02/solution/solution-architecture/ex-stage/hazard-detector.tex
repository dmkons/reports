The Hazard Detector looks at the instructions at different places in the pipeline and detects if there is a possibility for a data hazard.
The Hazard Detector is used by the forwarding unit to determine whether or not values should be forwarded.
The Hazard Detector does not recognize control hazards or use-after-load hazards.
This is because they are handled by other components of the system.

\subsubsection{In Signals}

\begin{description}
\item{\textbf{register\_address\_in}} \\
    The address of a register that is needed for an arbitrary execution.

\item{\textbf{register\_write\_execute\_in}} \\
    A boolean flag indicating whether the register destination from the execute stage will be written to.

\item{\textbf{register\_write\_memory\_in}} \\
    A boolean flag indicating whether the register destination from the memory stage will be written to.

\item{\textbf{register\_destination\_execute\_in}} \\
    The destination register of the value from the execution stage.
\item{\textbf{register\_destination\_memory\_in}} \\
    The destination register of the value from the execution stage.

\end{description}

\subsubsection{Out Signals}

\begin{description}
\item{\textbf{hazard\_out}} \\
    A two-bit status flag indicating whether or not there is risk of a data hazard. The two bits are used for designating if the data hazard originates in the execute stage or the memory stage.
\end{description}
