The ALU, or the arithmetic logic unit, is the heart of the processor.
The ALU is responsible for doing actual arithmetic and logical operations on data.
The rest of the processor is in reality machinery working to feed the ALU with as much data as quickly as possible.
It is therefore important that the ALU can work as fast as possible, in accordance to the performance design goal from section\vref{subsection:performance}.
The processor uses dedicated DSP\todo{what is DSP} slices on the FPGA to speed up the performance of the ALU.\todo{does it? have we disabled DSP slices in the synthesize menu?}

\subsubsection{In Signals}

\begin{description}
\item{\textbf{Y}} \\
The first operand of an ALU operation.

\item{\textbf{X}} \\
The second operand of an ALU operation.

\item{\textbf{Function}} \\
The function code that decides which operation the ALU should perform on $ X $ and $ Y $.
\end{description}

\subsubsection{Out Signals}

\begin{description}
\item{\textbf{Result}} \\
    The result of the ALU operation.

\item{\textbf{Zero}} \\
    A Boolean value which is set if the result of the ALU operation is 0, or unset if the result of the ALU operation is not zero.
    This signal is used for determining the outcome of comparisons, so that they may be used in conditionals.
\end{description}
