\section{Project Setup}

The Xilinx ISE project was set up with the design properties as in table \vref{table:design-properties}.

\begin{table}[H]
    \begin{center}
        \begin{tabular}{ | l | l | }
            \multicolumn{1}{ l }{\textbf{Property Name}} &
            \multicolumn{1}{ l }{\textbf{Value}} \\
            \multicolumn{2}{ l }{ } \\
            \hline
            Top-Level Source Type & HDL \\
            \hline
            Family & Spartan6 \\
            Device & XC6SLX16 \\
            Package & CSG324 \\
            Speed & -3 \\
            \hline
            Synthesis tool & XST (VHDL/Verilog) \\
            Simulator & ISim (VHDL/Verilog) \\
            Preferred Language & VHDL \\
            Property Specification in Project File & Store non-default values only \\
            Manual Compile Order & (not checked) \\
            VHDL Source Analysis Standard & VHDL-93 \\
            \hline
            Enable Message Filtering & (unchecked) \\
            \hline
        \end{tabular}
        \caption{Xilinx ISE Project Design Properties}
        \label{table:design-properties}
    \end{center}
\end{table}

\todo{Verify that this table is still correct}

\section{Tools}

This section contains a description list of the tools used for the assignment.

\todo{Add any more stuff here}

\subsection{Software}
\begin{description}
    \item{\textbf{ISE Project Navigator 12.4 (nt64) M.81d, expired licence}} \\
        Main IDE for writing VHDL.
    \item{\textbf{ISim 12.4 (nt64) M.81d, expired license}} \\
        Main simulation environment for simulating VHDL.
    \item{\textbf{ModelSim SE 6.6d}} \\
        Secondary simulation environment for simulating VHDL.
    \item{\textbf{Xilinx Platform Studio 12.4 (nt64) Build EDK\_MS4.81d+1, expired licence}} \\
        Used for preparing compiled VHDL for the FPGA board.
    \item{\textbf{Avnet Programming Utility}} \\
        Used for configuring the FPGA.
    \item{\textbf{Text editors}} \\
        Sublime Text 2, Vim 7.3, Notepad (©Copyright Microsoft Corporation).
    \item{\textbf{GHDL 0.29 (2010109) [Sokcho edition]}} \\
        Free open source VHDL simulator front end for GCC.

    \item{\textbf{GNU command-line tools}} \\
        Grep, sed, find, etc.
    \item{\textbf{git 1.8.1.2}} \\
        Version control system.
    \item{\textbf{GitHub}} \\
        Remote code repository hosting, issue tracking, wiki for logging.
\end{description}

\subsection{Hardware}
\begin{description}
\item{\textbf{Xilinx Spartan-6 XC6SLX16 CSG324}} \\
    FPGA board.
\item{\textbf{Development PC, Windows 7}} \\
    For development.
\item{\textbf{Mini USB cable}} \\
    For connecting the FPGA board to the development computer.
\end{description}


\subsection{Linux Tool Chain}

In addition to the Windows-based proprietary Xilinx tool chain available in the lab, a Linux-based free open source tool called GHDL was used.
GHDL is the leading open source VHDL compiler\cn, and offers many benefits over the clunky Xilinx tool chain for rapid development and iteration.
Although it is not as feature-rich as the Xilinx tools, and does not offer critical functionality such as synthesizing, it out-performs Xilinx in other areas.
Two areas in which it excels are intelligent error messages during analysis, and speed of compilation.
\newpage
As a comparison, consider a typical error message from Xilinx ISE in figure \vref{figure:xilinx-error-message}, and compare it to the error message produced by GHDL for the same error in \vref{figure:ghdl-error-message}.
This error was produced by removing the keyword \texttt{begin} from the architecture begin statement of the \texttt{flip\_flip} entity.
Especially dumbfounding is the error message in figure \vref{figure:xilinx-error-message} referring to an \texttt{is\_x}.
At no point in the VHDL code is there any string even remotely resembling \texttt{is\_x}.
The GHDL error message in figure \vref{figure:ghdl-error-message}, on the other hand, is short and concise, and actually tells the programmer exactly what is wrong, and what can be done to fix it.

\begin{figure}
\texttt{
flip\_flop.vhd:21:9: 'begin' is expected instead of 'elsif'
}
\begin{center}
\caption{A typical error message from GHDL.}
\label{figure:ghdl-error-message}
\end{center}
\end{figure}

\begin{figure}
\texttt{
Line 27: Syntax error near "flip\_flop".\\
Line 27: Expecting type  void for <flip\_flop>.\\
Line 29: Syntax error near "behavioral".\\
Line 29: Expecting type  void for <behavioral>.\\
Line 32: Expecting type  void for <ieee>.\\
Line 33: is\_x is not a  void\\
Line 33: Formal <s> has no actual or default value.\\
s is declared here\\
Line 35: Syntax error near "entity".\\
...\\
}
\begin{center}
\caption{A typical error message from Xilinx ISE.}
\label{figure:xilinx-error-message}
\end{center}
\end{figure}
