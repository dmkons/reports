This chapter discusses the performance and test results of the presented solution processor, as well as the assignment itself and the work process around it.

\section{Requirements}

The requirements stated that an optimized pipeline processor be implemented in VHDL and programmed onto an FPGA.
The solution processor presented in this report meets these requirements.

An instruction set roughly twice the size of the minimal required instruction set is supported by the solution presented in this report, and all the required instructions are of course supported.

The processor is highly optimized, implementing features such as data forwarding and branch prediction to achieve great performance.
To put things in perspective, the processor in this assignment is capable of executing over 137 million instructions per second in optimal conditions.
This is an order of magnitude faster than the simple multi-cycle processor presented in \cite{assignment-1}.

\section{Additional Goals}

The first of the two additional goals for this assignment was simplicity, as documented in section \vref{subsection:simplicity}.
The presented solution processor implements a considerably less obese instruction set compared to the processor in \cite{assignment-1}, and is implementation-wise less complex in many areas.
Because of this, the solution processor has achieved a better test coverage than the processor presented in \cite{assignment-1}, and it has been successfully programmed onto an FPGA.

The second of the two additional goals for this assignment was performance, as documented in section \vref{subsection:performance}.
The presented solution processor is a performant processor, so this additional goal was also met to a satisfactory degree.

Focus on energy efficiency for the solution processor was neither a requirement nor an additional goal.
Still, some measurements were made for the sake of completeness.
As is readily apparent from the measurements in \vref{results:energy-efficiency}, the processor is reasonably energy efficient, with the quiescent power usage of the FPGA board dwarfing the power usage of that of the processor itself.

\section{About the Work Process}

Time management is the strongest skill asset attributable to the authors of this report.
Much like the work process illustrated in\cite{assignment-1}, most of the work for this assignment was completed near the deadline.
Additionally, circumstances should have it that other academic pursuits, namely the computer project in TDT4295, a course offered by the same faculty that offers the course for which this report is written, have overshadowed this assignment.
Coinciding deadlines in both courses have been impractical, but in the end late nights and long hours compensated for the lack of co-ordination.
