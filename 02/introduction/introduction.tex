This report presents a solution to ``Assignment \#2 - An Optimized Pipelined Processor'' of the autumn 2013 course TDT4255 Computer Design at NTNU.

\section{Assignment}

The assignment is to implement an optimized pipelined processor in VHDL, and implement it on a Xilinx Spartan 6 FPGA.
The processor must be performance-optimized using hazard detection and correction techniques.

\subsection{Requirements}

The major requirement of the assignment is a simple 5-staged pipelined processor\cn.
Additionally, hazard detection and correction techniques must be implemented.

\section{Goals}

In addition to the requirements set forth by the assignment, some additional design goals have been decided upon.
These goals have been chosen based on what went well, and what could have gone better in the previous assignment\cn.

\subsection{Simplicity}

Reflecting on the previous assignment\cn, the biggest flaw was that the processor was never successfully run on a physical FPGA.
In retrospect, this was because a too large feature scope was selected, which resulting in there not being enough time to compile the processor onto the FPGA after its completion.
For this assignment, simplicity is a design goal in the hope that there will be enough time to complete physical testing of the processor on an FPGA.

\subsection{Performance}

\todo{this}
