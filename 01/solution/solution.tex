Describes your solution of the task.
Contains a detailed description of all the subtasks which have been solved and how they contribute to the solution for the given task.
The use of diagrams, figures, tables and similar is welcome as a support to your description.

The processor presented as a solution for assignment 1 in this report is a 32-bit simple MIPS-inspired multi-cycle processor described in VHDL and programmed onto a Xilinx Spartan-6 LX25 FPGA.

\section{Requirements}

The requirements for the assignment is to design a simple multi-cycle MIPS architecture.
It must be based around the suggested MIPS-like architecture in Figure 4.1 of the course compendium\cite{compendium}.
Minimally, the 10 insructions in table \ref{table:required-instructions} must be implemented.

\begin{table}
    \begin{center}
        \begin{tabular}{r|l}
            \texttt{ADD} & Add \\
            \texttt{AND} & And \\
            \texttt{BEQ} & Branch if equal \\
            \texttt{J} & Jump \\
            \texttt{LDI}\footnote{\texttt{LDI} is not actually a MIPS instruction, but it is assumed that it is interchangeable with \texttt{LUI}} & Load immediate \\
            \texttt{LW} & Load word \\
            \texttt{OR} & Or \\
            \texttt{SLT} & Set less than \\
            \texttt{SUB} & Subtract \\
            \texttt{SW} & Store word \\
            \hline
        \end{tabular}
        \smallskip
        \smallskip
        \caption{Required instructions}
        \label{table:required-instructions}
    \end{center}
\end{table}

Optionally, more instructions can be implemented.
The instruction set implemented by the solution processor is detailed in section [THE SECTION ABOUT THE ISA].

Once the processor is designed, it must also be verified in a simulation environment as well as in a hardware platform.

TODO: cite the compendium well enough when there is talk about requirements.

\section{Solution Architecture}

TODO: an image, box and arrows-style

\subsection{ALU}

Something about the alu.

\subsection{Control Unit}

The control unit is a finite state machine that controls the processor.
Its job is to maintain control state across cycles, and distribute the correct control signals to the other components in the processor at the right time.
The finite state machine has three states: fetch, execute and stall.
It advances to the next state on every cycle, with transitions as illustrated in figure \ref{figure:control-unit-state-machine}.

\begin{figure}[h]
    \begin{center}
        \begin{tikzpicture}[->,>=stealth',shorten >=1pt,auto,node distance=2.8cm, semithick]
            \tikzstyle{every state}=[fill=none,draw=black,text=black]
            \node[initial,state] (fetch)                    {$fetch$};
            \node[state]         (execute) [right of=fetch] {$execute$};
            \node[state]         (stall) [right of=execute] {$stall$};
            \path (fetch) edge node {} (execute)
                (execute) edge [bend left] node {non-mem} (fetch)
                (execute) edge node {mem} (stall)
                (stall) edge [bend right] node {} (fetch);
        \end{tikzpicture}
            \caption{
                Control unit state machine.
                \texttt{mem} transitions are taken when the currently executed instruction accesses memory.
            }
            \label{figure:control-unit-state-machine}
    \end{center}
\end{figure}

TODO: explain the signals that come out of the control unit

\subsection{Program Counter Cycle}

The program counter cycle is the loop the program counter forms together with the branch and jump circuitry.
The program counter is a D-latch register (TODO: is it?) that holds the address of the next instruction to be fetched.
When no branching or jumping is involved (the "sunny day scenario"), the program counter cycle increments the program counter by one every cycle the control unit tells it to.
This means that the control unit has the opportunity to advance the running program.
The control unit advances the program counter when it is in the execute state, so that a new program counter value is ready for when the control unit enters the fetch state.

When the control unit gives the signal, the program counter may perform a branch or a jump by manipulating the program counter cycle to change the value sent back to the program counter.

TODO: diagram of the program counter cycle.

\subsection{Multiplexer}

The multiplexer... do we even need to talk about it?

\section{Instruction Set Architecture}

The solution processor implements a modified subset of the MIPS instruction set.
A quick reference of the MIPS instruction set can be found in Figure 3.4 of the compendium \cite{compendium}.
The instructions, as in regular MIPS, can be on one of three general formats, R, I and J:


\bigskip

\begin{center}
    \begin{bytefield}[endianness=big,bitwidth=0.03125\linewidth]{32}
        \bitheader{0-31} \\
        \begin{rightwordgroup}{R}
            \bitbox{6}{Opcode} &
            \bitbox{5}{Rs} &
            \bitbox{5}{Rt} &
            \bitbox{5}{Rd} &
            \bitbox{5}{Shift}
            \bitbox{6}{Function}
        \end{rightwordgroup} \\
    \end{bytefield}
\end{center}

\bigskip

\begin{center}
    \begin{bytefield}[endianness=big,bitwidth=0.03125\linewidth]{32}
        \bitheader{0-31} \\
        \begin{rightwordgroup}{I}
            \bitbox{6}{Opcode} &
            \bitbox{5}{Rs} &
            \bitbox{5}{Rt} &
            \bitbox{16}{Immediate}
        \end{rightwordgroup} \\
    \end{bytefield}
\end{center}

\bigskip

\begin{center}
    \begin{bytefield}[endianness=big,bitwidth=0.03125\linewidth]{32}
        \bitheader{0-31} \\
        \begin{rightwordgroup}{J}
            \bitbox{6}{Opcode} &
            \bitbox{26}{Address}
        \end{rightwordgroup} \\
    \end{bytefield}
\end{center}

(Yes, I just included these diagrams to show off with latex, really we can just refer to the reference sheet in the compendium).

The solution processor implements the instructions in table \ref{table:implemented-instructions}.
The processor supports quite a few more instructions than the minimum requirements.
This is done because (TODO: convincing argument).

TODO: alphabetize \ref{table:implemented-instructions}

\begin{table}
    \begin{center}
        \begin{tabular}{r|l}
            \texttt{ADDI} & Add immediate \\
            \texttt{ADDIU} & Add immediate unsigned \\
            \texttt{ANDI} & And immediate \\
            \texttt{BNE} & Branch not equal \\
            \texttt{LUI} & Load upper immediate \\
            \texttt{LW} & Load word \\
            \texttt{ORI} & Or immediate \\
            \texttt{SLTI} & Set less than immediate \\
            \texttt{SLTIU} & Set less than immediate unsigned \\
            \texttt{SW} & Store word \\
            \texttt{XORI} & Xor immediate \\
            \texttt{J} & Jump \\
            \texttt{ADD} & Add \\
            \texttt{ADDU} & Add unsigned \\
            \texttt{AND} & And \\
            \texttt{MULT} & Multiply \\
            \texttt{MULTU} & Multiply unsigned \\
            \texttt{NOR} & Nor \\
            \texttt{OR} & Or \\
            \texttt{SLL} & Shift left logical \\
            \texttt{BEQ} & Branch if equal \\
            \texttt{SLLV} & Shift left logical variable \\
            \texttt{SLT} & Set less than \\
            \texttt{SLTU} & Set less than unsigned \\
            \texttt{SRA} & Shift right arithmetic \\
            \texttt{SRAV} & Shift right arithmetic variable \\
            \texttt{SRL} & Shift right logical \\
            \texttt{SRLV} & Shift right logical variable \\
            \texttt{SUB} & Subtract \\
            \texttt{SUBU} & Subtract unsigned \\
            \texttt{XOR} & Xor \\
            \texttt{PASSTHROUGH} & Passthrough (i.e. send the first input through unmodified) \\
        \end{tabular}
        \smallskip
        \hrule
        \smallskip
        \caption{Implemented instructions}
        \label{table:implemented-instructions}
    \end{center}
\end{table}

\section{Deliverables}

This is where we list all the deliverables we're sending in.
