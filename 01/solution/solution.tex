Describes your solution of the task.
Contains a detailed description of all the subtasks which have been solved and how they contribute to the solution for the given task.
The use of diagrams, figures, tables and similar is welcome as a support to your description.

The processor presented as a solution for assignment 1 in this report is a 32-bit simple MIPS-inspired multi-cycle processor described in VHDL and programmed onto a Xilinx Spartan-6 LX25 FPGA.

\section{Solution Architecture}

\todo{Generally describe the solution architecture}

\todo{Solution architecture image}

The rest of this section describes the different architectural subcomponents in detail.

\todo{detailed description of the components}

\subsection{ALU}

The ALU, or the arithmetic logic unit, is the heart of the processor.
The ALU is responsible for doing actual arithmetic and logical operations on data.
The rest of the processor is in reality machinery working to feed the ALU with as much data as quickly as possible.
It is therefore important that the ALU can work as fast as possible, in accordance to the performance design goal from section\vref{subsection:performance}.
The processor uses dedicated DSP slices on the FPGA to speed up the performance of the ALU.

\subsubsection{In Signals}

\begin{description}
\item{\textbf{Y}} \\
The first operand of an ALU operation.

\item{\textbf{X}} \\
The second operand of an ALU operation.

\item{\textbf{Function}} \\
The function code that decides which operation the ALU should perform on $ X $ and $ Y $.
\end{description}

\subsubsection{Out Signals}

\begin{description}
\item{\textbf{Result}} \\
    The result of the ALU operation.

\item{\textbf{Zero}} \\
    A Boolean value which is set if the result of the ALU operation is 0, or unset if the result of the ALU operation is not zero.
    This signal is used for determining the outcome of comparisons, so that they may be used in conditionals.
\end{description}

\newpage



\subsection{ALU}

Something about the alu.

\subsection{Branch Controller}

The branch controller the logic unit that decide whether or not the program should branch.
It is seperated from the main controller unit, as it works independent from the state machine, and does not need to follow the clock.
The branch controller read the op code field from the instruction, and if it is a branch instruction it will run the logic that is requiered for that instruction.

To handle the simplest operations that compare two registers to each other, the branch controller can simply look at the flags from the ALU to decide if the branch mux should be enabled.
Instructions that compare a register to zero require a bit more work from the branch controller.
The branch controller can send out a zero value that overrides the alus source for the y operand, and is compared to the x operand from the register specified in the instruction.

To allow for other compare operations than $x \geq 0$ and $x < 0$, the zero vaule will not always be zero. 
Because our CPU only handle integers, we can say that $x \geq 1$ is equivalent with $x > 0$.
This trick mean that we do not need to support more than zero and negative, and can save quite a bit of complexity in our design without loosing performance.
Therefore the zero value from the branch controller vary from -1 to 1.

The control unit will assist the branch controller by instructing the ALU to do a subtraction on its inputs, x - y, and discard the result.
This will set the zero flag of the ALU if the x = y, and the negative flag of the ALU if the x < y.
It is these flags that is used by the branch controller to detemin if it should branch.


\subsection{Control Unit}

The control unit is a finite state machine that controls the processor.
Its job is to maintain control state across cycles, and distribute the correct control signals to the other components in the processor at the right time.
The finite state machine has three states: fetch, execute and stall.
It advances to the next state on every cycle, with transitions as illustrated in figure \ref{figure:control-unit-state-machine}.

\begin{figure}[h]
    \begin{center}
        \begin{tikzpicture}[->,>=stealth',shorten >=1pt,auto,node distance=2.8cm, semithick]
            \tikzstyle{every state}=[fill=none,draw=black,text=black]
            \node[initial,state] (fetch)                    {$fetch$};
            \node[state]         (execute) [right of=fetch] {$execute$};
            \node[state]         (stall) [right of=execute] {$stall$};
            \path (fetch) edge node {} (execute)
                (execute) edge [bend left] node {non-mem} (fetch)
                (execute) edge node {mem} (stall)
                (stall) edge [bend right] node {} (fetch);
        \end{tikzpicture}
            \caption{
                Control unit state machine.
                \texttt{mem} transitions are taken when the currently executed instruction accesses memory.
            }
            \label{figure:control-unit-state-machine}
    \end{center}
\end{figure}

TODO: explain the signals that come out of the control unit

\subsection{Program Counter Cycle}

The program counter cycle is the loop the program counter forms together with the branch and jump circuitry.
The program counter is a D-latch register (TODO: is it?) that holds the address of the next instruction to be fetched.
When no branching or jumping is involved (the "sunny day scenario"), the program counter cycle increments the program counter by one every cycle the control unit tells it to.
This means that the control unit has the opportunity to advance the running program.
The control unit advances the program counter when it is in the execute state, so that a new program counter value is ready for when the control unit enters the fetch state.

When the control unit gives the signal, the program counter may perform a branch or a jump by manipulating the program counter cycle to change the value sent back to the program counter.

TODO: diagram of the program counter cycle.

\subsection{Multiplexer}

The multiplexer... do we even need to talk about it?

\section{Instruction Set}

The solution processor implements a modified subset of the MIPS instruction set.
A quick reference of the MIPS instruction set can be found in Figure 3.4 of the compendium \cite{compendium}.
The instructions, as in regular MIPS, can be on one of three general formats, R, I and J.

The solution processor implements the instructions in table \ref{table:implemented-instructions}.
The processor supports quite a few more instructions than the minimum requirements.
This is done because (TODO: convincing argument).

\begin{table}
    \begin{center}
        \begin{tabular}{r|l}
            \texttt{ADD} & Add \\
            \texttt{ADDI} & Add immediate \\
            \texttt{ADDIU} & Add immediate unsigned \\
            \texttt{ADDU} & Add unsigned \\
            \texttt{AND} & And \\
            \texttt{ANDI} & And immediate \\
            \texttt{BEQ} & Branch if equal \\
            \texttt{BNE} & Branch not equal \\
            \texttt{J} & Jump \\
            \texttt{LUI} & Load upper immediate \\
            \texttt{LW} & Load word \\
            \texttt{MULT} & Multiply \\
            \texttt{MULTU} & Multiply unsigned \\
            \texttt{NOR} & Nor \\
            \texttt{OR} & Or \\
            \texttt{ORI} & Or immediate \\
            \texttt{PASSTHROUGH} & Passthrough (i.e. send the first input through unmodified) \\
            \texttt{SLL} & Shift left logical \\
            \texttt{SLLV} & Shift left logical variable \\
            \texttt{SLT} & Set less than \\
            \texttt{SLTI} & Set less than immediate \\
            \texttt{SLTIU} & Set less than immediate unsigned \\
            \texttt{SLTU} & Set less than unsigned \\
            \texttt{SRA} & Shift right arithmetic \\
            \texttt{SRAV} & Shift right arithmetic variable \\
            \texttt{SRL} & Shift right logical \\
            \texttt{SRLV} & Shift right logical variable \\
            \texttt{SUB} & Subtract \\
            \texttt{SUBU} & Subtract unsigned \\
            \texttt{SW} & Store word \\
            \texttt{XOR} & Xor \\
            \texttt{XORI} & Xor immediate \\
        \end{tabular}
        \smallskip
        \hrule
        \smallskip
        \caption{Implemented instructions}
        \label{table:implemented-instructions}
    \end{center}
\end{table}

\section{Test utilities}

TODO: move the solution component description of the test utilities from results-and-tests here

\section{Deliverables}

This is where we list all the deliverables we're sending in.
