VHDL supports a basic \texttt{assert} statement that can be used to programatically compare simulated values against expected results, and with the optional \texttt{report} keyword, additionally print a string to the console if the assertion fails.
VHDL's support for automatic non-data-driven unit testing beyond the basic \texttt{assert} statement is rather lacking.
Specifically, it shows neither the expected value of the assertion, nor the actual value found.
Because of this, a VHDL package called \texttt{test\_utils} was written.
It contains a convenient testing function, creatively called \texttt{test}.
It supports printing expected values and actual found values, as well as a simple tagging and naming system.
The latter makes it easier to give meaningful names to the different asserts in a test bench.
