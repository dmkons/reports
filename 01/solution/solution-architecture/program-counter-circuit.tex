The program counter circuit is the loop the program counter forms together with the branch and jump circuitry.
The program counter is a register that holds the address of the next instruction to be fetched.
When no branching or jumping is involved (the ``sunny day scenario''), the program counter cycle increments the program counter by one every time the control unit tells it to.
This means that the control unit has the opportunity to advance the running program.
The control unit advances the program counter when it is in the execute state, so that a new program counter value is ready for when the control unit enters the fetch state.

When the control unit gives the signal, the program counter may perform a branch or a jump by manipulating the program counter circuit to change the value sent back to the program counter.

TODO: diagram of the program counter cycle.

