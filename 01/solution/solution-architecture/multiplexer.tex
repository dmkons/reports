The multiplexer is a simple, ordinary multiplexer implemented as a VHDL entity for tidiness and modularity.
Although quite trivial, it is documented here for completeness.

\subsubsection{In Signals}

\begin{description}
\item{\textbf{Input 0}} \\
A generic \texttt{std\_logic\_vector} input.

\item{\textbf{Input 1}} \\
A generic \texttt{std\_logic\_vector} input.

\item{\textbf{Selector}} \\
A boolean select signal that selects either Input 0 or Input 1 to be passed on as output.
In the architecture diagram these are labeled as ``\texttt{ctrl}''. 
In the VHDL code it is the ``\texttt{enable}'' signal.

\end{description}

\subsubsection{Out Signals}

\begin{description}
\item{\textbf{Output}} \\
Input 0 or Input 1, decided by the Selector.
\end{description}
