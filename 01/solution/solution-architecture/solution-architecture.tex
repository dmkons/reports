The solution architecture is based on the suggested architecture in Figure 4.1 of the compendium~\cite[p.115]{compendium}.
In order to support a larger instruction set, the solution architecture has been expanded somewhat.
The main differences are that the ALU control unit and the main control unit have been merged, and that a branch controller has been added to take care of branching logic.
The architecture of the presented solution processor is illustrated in figure \vref{figure:cpu-architecture}.

In figure \vref{figure:cpu-architecture}, each box corresponds to an architecture component.
Larger components have their names written the top of each box.
Smaller components have hovering labels showing their names.
The left side of a box is used for input signals, and the right side of the box is used for output signals.
The exception to this rule are the muxes, which use a variant of the classic pill-style mux diagrams: Input signals come through the top and bottom, the control signal comes in on the middle left and the output is on the middle right.

The rest of this section describes the different architectural subcomponents in detail.

\begin{sidewaysfigure}[H]
	\begin{center}
		\includegraphics[keepaspectratio, height=\textheight, width=\textwidth]{graphics/cpu-architecture/cpu-architecture-color.pdf}
		\caption{Architecture of the solution processor.}
		\label{figure:cpu-architecture}
	\end{center}
\end{sidewaysfigure}

\subsection{ALU}

The ALU, or the arithmetic logic unit, is the heart of the processor.
The ALU is responsible for doing actual arithmetic and logical operations on data.
The rest of the processor is in reality machinery working to feed the ALU with as much data as quickly as possible.
It is therefore important that the ALU can work as fast as possible, in accordance to the performance design goal from section\vref{subsection:performance}.
The processor uses dedicated DSP slices on the FPGA to speed up the performance of the ALU.

\subsubsection{In Signals}

\begin{description}
\item{\textbf{Y}} \\
The first operand of an ALU operation.

\item{\textbf{X}} \\
The second operand of an ALU operation.

\item{\textbf{Function}} \\
The function code that decides which operation the ALU should perform on $ X $ and $ Y $.
\end{description}

\subsubsection{Out Signals}

\begin{description}
\item{\textbf{Result}} \\
    The result of the ALU operation.

\item{\textbf{Zero}} \\
    A Boolean value which is set if the result of the ALU operation is 0, or unset if the result of the ALU operation is not zero.
    This signal is used for determining the outcome of comparisons, so that they may be used in conditionals.
\end{description}

\newpage

\subsection{Branch Controller}

The branch controller the logic unit that decides whether or not the program should branch.
It is separate from the main control unit, as it works independently from the control unit state machine, and does not need to follow the clock.
The branch controller reads the opcode field from the current instruction, looking for branch instructions.
If it finds a branch instruction, it will apply the logic that is required for that instruction.

Having branch logic contained in a separate branch controller gives the opportunity to implement many different branch instructions in a modular manner.

To handle the simplest operations that compare two registers to each other, the branch controller can simply look at the flags from the ALU to decide if the branch MUX should be enabled.
Instructions that compare a register to zero require a bit more work from the branch controller.
The branch controller can send out a zero value that overrides the ALU's source for the $ y $ operand, and is compared to the $ x $ operand from the register specified in the instruction.

To allow for other compare operations than $x \geq 0$ and $x < 0$, the zero value will not always be zero. 
Because the solution processor only handles integers, the comparison $ x \geq 1 $ is equivalent with $ x > 0 $.
This trick means that support for branch operations other than zero and negative are not needed, which nicely reduces complexity in the design without sacrificing performance.
Therefore the zero value from the branch controller vary from -1 to 1.

The control unit drives the branch controller by instructing the ALU to do a subtraction on its inputs, $ x - y $, and ignore the result.
This sets the zero flag of the ALU if the x = y, and the negative flag of the ALU if the x < y.
It is these flags that is used by the branch controller to detemin if it should branch.

\newpage

\subsection{Control Unit}

The control unit is the component that is responsible for enabling and disabling the correct parts of the processor at the correct times, so that an instruction is executed correctly.

\subsubsection{In Signals}

\begin{description}
\item{\textbf{Instruction Op-code}} \\
    The op-code of the currently executing instruction in the instruction decode stage.

\item{\textbf{Instruction Function}} \\
    The ALU function of the currently executing instruction in the instruction decode stage. 
\end{description}

\subsubsection{Out Signals}

\begin{description}
\item{\textbf{Execute Control Signals}} \\
    The control signals that should be used in the execute stage for the instruction being decoded by the control unit.
    The execute control signals bus contains the \textbf{ALU Source}, \textbf{ALU Function} and \textbf{Register Destination} control signals.

\item{\textbf{Memory Control Signals}} \\
    The control signals that should be used in the memory stage for the instruction being decoded by the control unit.
    The memory control signals bus contains the \textbf{Branch}, \textbf{Jump} and \textbf{Memory Write} control signals.

\item{\textbf{Write-Back Control Signals}} \\
    The control signals that should be used in the write-back stage for the instruction being decoded by the control unit.
The write-back control signals bus contains the \textbf{Memory to Register} and \textbf{Register Write} control signals.
\end{description}


\newpage

\subsection{Program Counter Circuit}

The ``program counter circuit'' is the loop formed by the Program Counter along with the Branch and Jump circuitry, see figure \vref{figure:pc-circuit}.

The program counter is a register that holds the address of the next instruction to be fetched.
When no branching or jumping is involved (the ``sunny day scenario''), the program counter cycle updates the program counter when the control unit tells it to.
This means that the control unit is the entity that advances the running program.
The control unit advances the program counter when it is in the execute state, so that a new program counter value is ready for when the control unit enters the fetch state.

When the program counter is updated the Program Counter Out signal is set to equal the Program Counter In signal.

When the control unit gives the signal, a branch or jump is performed by manipulating the program counter circuit to change the value sent back to the program counter.

\begin{figure}[h!]
	\begin{center}
		\includegraphics[keepaspectratio, height=\textheight, width=\textwidth]{graphics/cpu-architecture/cpu-arc-pc-circuit.pdf}
		\caption{Architecture of the solution processor. The signals and components directly involved with the program counter circuit are highlighted in green.}
		\label{figure:pc-circuit}
	\end{center}
\end{figure}


\subsubsection{Program Counter}

The program counter is implemented as a D-type flip-flop with reset and enable signals.

\subsubsubsection{In Signals}

\begin{description}
\item{\textbf{Clock}} \\

Clock signal.

\item{\textbf{Program Counter In}} \\

The Program Counter's next value.

\item{\textbf{Reset}} \\

When the reset signal is high, the program counter is reset to $-1$\footnote{\texttt{FF FF FF FF}} on the next rising edge of the clock.

\item{\textbf{Program Counter Enable}} \\

Tells the Program Counter whether or not the Program Counter should be updated.
When this signal is \texttt{1} the Program Counter is updated.

\end{description}

\subsubsubsection{Out Signals}

\begin{description}
\item{\textbf{Program Counter Out}} \\

The Program Counter Out signal is the address of the current instruction.

\end{description}

For more information about how the program counter circuit works, refer to figure \vref{figure:pc-circuit}.

\newpage

\subsection{Multiplexer}

The multiplexer is a simple regular multiplexer implemented as a VHDL entity, for tidyness and modularity.
Although quite trivial, it is documented here for completeness.

\subsubsection{In Signals}

\begin{description}
\item{\textbf{Input 0}} \\

A generic \texttt{std\_logic\_vector} input.

\item{\textbf{Input 1}} \\

A generic \texttt{std\_logic\_vector} input.

\item{\textbf{Selector}} \\

A boolean select signal that selects either Input 0 or Input 1 to be passed on as output.

\end{description}

\subsubsection{Out Signals}

\begin{description}
\item{\textbf{Output}} \\

Input 0 or Input 1, decided by the Selector.

\end{description}

\newpage
