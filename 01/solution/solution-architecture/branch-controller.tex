The branch controller is the logic unit that decides whether or not the program should branch.
It is separate from the main control unit, as it works independently from the control unit state machine, and does not need to follow the clock.
The branch controller reads the opcode field from the current instruction, looking for branch instructions.
If it finds a branch instruction, it will apply the logic that is required for that instruction.

Having branch logic contained in a separate branch controller gives the opportunity to implement many different branch instructions in a modular manner.

To handle the simplest operations that compare two registers to each other, the branch controller can simply look at the flags from the ALU to decide if the branch MUX should be enabled.
Instructions that compare a register to zero require a bit more work from the branch controller.
The branch controller can send out a zero value that overrides the ALU's source for the $ y $ operand, and is compared to the $ x $ operand from the register specified in the instruction.

To allow for other compare operations than $x \geq 0$ and $x < 0$, the zero value will not always be zero. 
Because the solution processor only handles integers, the comparison $ x \geq 1 $ is equivalent with $ x > 0 $.
This trick means that support for branch operations other than zero and negative are not needed, which nicely reduces complexity in the design without sacrificing performance.
Therefore the zero value from the branch controller vary from -1 to 1.

The control unit drives the branch controller by instructing the ALU to do a subtraction on its inputs, $ x - y $, and ignore the result.
This sets the zero flag of the ALU if $x = y$, and the negative flag of the ALU if $x < y$.
These flags are used by the branch controller to determine if it should branch.

\subsubsection{In Signals}

\begin{description}
\item{\textbf{Flags}} \\

The Flags signal comes directly from the ALU, and it tells the branch controller the outcome of a compare comparison.

\item{\textbf{Opcode}} \\

The opcode signal comes directly from the currently executing instruction word and tells the branch controller what kind of branch it should check for, if any. The opcode signal consists of bits 31 through 26 of the current instruction.

\end{description}

\subsubsection{Out Signals}

\begin{description}
\item{\textbf{Compare Zero Value}} \\

Sometimes it is neccessary to compare to a value of zero that is actually in the range $[-1,1]$.
This zero value is then put here on this bus.

\item{\textbf{Compare Zero}} \\

If the compare operation will compare some value x to 0, this signal will override the y operand with the value from compare\_zero\_value.

\item{\textbf{Branch}} \\

If the branch controller decides that the program should branch, the Branch signal is set and the next instruction address becomes the current PC+1 added together with the offset instead of PC+1.
In MIPS, the offset is the lower 16 bits on all branch instructions.

\end{description}
