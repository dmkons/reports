Introduces the task of the assignment and the challenges it
brings.
Also, it gives a brief introduction to how the task was approached and in which way the solution was reached.


This report presents a solution to ``Assignment \#1 - Simple Multi-cycle MIPS Processor'' of the autumn 2013 course TDT4255 at NTNU.

\section{Assignment}

A summery of the assignment can be found in section 4.1 of the compendium~\cite[p.114]{compendium}, and is reproduced below for convenience:

\begin{quote}
"In this assignment, you will design a simple multi-cycle MIPS processor in VHDL and synthesize your design by following the procedure described in Chapter 2~\cite[of the compendium]{compendium}.
You will also verify the behaviour of the implemented MIPS processor using the ModelSim simulator.
Once your design is verified, you will integrate the MIPS processor into a MicroBlaze-based embedded system as a peripheral core, implement the embedded system design in FPGA\footnote{It was assumed that the assignment author means "VHDL" here} and the designed processor in an FPGA."
\end{quote}

\subsection{Requirements}

The assignment's main requirement is to design a ``simple multi-cycle MIPS architecture''~\cite[p.114]{compendium}. It must be based on the suggested MIPS-like architecture presented in Figure 4.1~\cite[p.115]{compendium} in the course compendium.

Minimally, the instructions from a given set of instruction classes have to be implemented. 
Examination of the list yielded 10 instructions that were deemed a part of the minimum requirement.
These are listed in Table \ref{table:required-instructions}.

\begin{table}
    \begin{center}
        \begin{tabular}{r|l}
            \texttt{ADD} & Add \\
            \texttt{AND} & And \\
            \texttt{BEQ} & Branch if equal \\
            \texttt{J} & Jump \\
            \texttt{LDI}/\texttt{LLI} & Load (lower) immediate \\
            \texttt{LW} & Load word \\
            \texttt{OR} & Or \\
            \texttt{SLT} & Set less than \\
            \texttt{SUB} & Subtract \\
            \texttt{SW} & Store word \\
            \hline
        \end{tabular}
        \smallskip
        \smallskip
        \caption{Required instructions.}
        \label{table:required-instructions}
    \end{center}
\end{table}

Note: the assignment text required the \texttt{LDI} (Load Immediate) instruction to be implemented.
However in MIPS, \texttt{LDI} is not an instruction.
The closest match in MIPS is the instruction called \texttt{LUI} (Load Upper Immedate).
This instruction loads an immediate, but also shifts it to put it in the upper half of the destination register.
As LUI does not do exactly what LDI would have done, a new instruction \texttt{LLI} (Load Lower Immediate), takes \texttt{LDI}'s place in the instruction requirements table.

Implementing more instructions than the aformentioned is optional.
The instruction set implemented by the solution processor is detailed in section [THE SECTION ABOUT THE ISA].

Once the processor is designed, it must also be verified in a simulation environment as well as in a hardware platform.
