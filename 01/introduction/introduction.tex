Introduces the task of the assignment and the challenges it
brings.
Also, it gives a brief introduction to how the task was approached and in which way the solution was reached.


This report presents a solution to "Assignment \#1 - Simple Multi-cycle MIPS Processor" of the autumn 2013 course TDT4255 at NTNU.

\section{Assignment}

A summery of the assignment can be found in section 4.1 of the compendium\cite{compendium}, and is reproduced below for convenience:

\begin{quote}
"In this assignment, you will design a simple multi-cycle MIPS processor in VHDL and synthesize your design by following the procedure described in Chapter 2 [of the compendium\cite{compendium}].
You will also verify the behaviour of the implemented MIPS processor using the ModelSim simulator.
Once your design is verified, you will integrate the MIPS processor into a MicroBlaze-based embedded system as a peripheral core, implement the embedded system design in FPGA\footnote{It is assumed that the assignment author means "VHDL" here} and the designed processor in an FPGA."
\end{quote}

\subsection{Requirements}

The requirements for the assignment is to design a simple multi-cycle MIPS architecture.
It must be based around the suggested MIPS-like architecture in Figure 4.1 of the course compendium\cite{compendium}.
Minimally, the 10 insructions in table \ref{table:required-instructions} must be implemented.

\begin{table}
    \begin{center}
        \begin{tabular}{r|l}
            \texttt{ADD} & Add \\
            \texttt{AND} & And \\
            \texttt{BEQ} & Branch if equal \\
            \texttt{J} & Jump \\
            \texttt{LDI}\footnote{\texttt{LDI} is not actually a MIPS instruction, but it is assumed that it is interchangeable with \texttt{LUI}} & Load immediate \\
            \texttt{LW} & Load word \\
            \texttt{OR} & Or \\
            \texttt{SLT} & Set less than \\
            \texttt{SUB} & Subtract \\
            \texttt{SW} & Store word \\
            \hline
        \end{tabular}
        \smallskip
        \smallskip
        \caption{Required instructions}
        \label{table:required-instructions}
    \end{center}
\end{table}

Optionally, more instructions can be implemented.
The instruction set implemented by the solution processor is detailed in section [THE SECTION ABOUT THE ISA].

Once the processor is designed, it must also be verified in a simulation environment as well as in a hardware platform.

TODO: cite the compendium well enough when there is talk about requirements.
