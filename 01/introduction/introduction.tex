This report presents a solution to ``Assignment \#1 - Simple Multi-cycle MIPS Processor'' of the autumn 2013 course TDT4255 at NTNU.

\section{Assignment}

A summery of the assignment can be found in section 4.1 of the compendium~\cite[p.114]{compendium}, and is reproduced below for convenience:

\begin{quote}
``In this assignment, you will design a simple multi-cycle MIPS processor in VHDL and synthesize your design by following the procedure described in Chapter 2~\cite[of the compendium]{compendium}.
You will also verify the behaviour of the implemented MIPS processor using the ModelSim simulator.
Once your design is verified, you will integrate the MIPS processor into a MicroBlaze-based embedded system as a peripheral core, implement the embedded system design in FPGA\footnote{It is assumed that the assignment author means "VHDL" here} and the designed processor in an FPGA.''
\end{quote}

\subsection{Requirements}

The assignment's main requirement is to design a ``simple multi-cycle MIPS architecture''~\cite[p.114]{compendium}. It must be based on the suggested MIPS-like architecture presented in Figure 4.1~\cite[p.115]{compendium} in the course compendium.

Minimally, the instructions from a given set of instruction classes have to be implemented. 
Examination of the list yielded 10 instructions that were deemed a part of the minimum requirement.
These are listed in Table \vref{table:required-instructions}.

\begin{table}
    \begin{center}
        \begin{tabular}{r|l}
            \texttt{ADD} & Add \\
            \texttt{AND} & And \\
            \texttt{BEQ} & Branch if equal \\
            \texttt{J} & Jump \\
            \texttt{LDI}/\texttt{LLI} & Load (lower) immediate \\
            \texttt{LW} & Load word \\
            \texttt{OR} & Or \\
            \texttt{SLT} & Set less than \\
            \texttt{SUB} & Subtract \\
            \texttt{SW} & Store word \\
            \hline
        \end{tabular}
        \smallskip
        \smallskip
        \caption{Required instructions.}
        \label{table:required-instructions}
    \end{center}
\end{table}

Note: the assignment text required the \texttt{LDI} (Load Immediate) instruction to be implemented.
However in MIPS, \texttt{LDI} is not an instruction.
The closest match in MIPS is the instruction called \texttt{LUI} (Load Upper Immedate).
This instruction loads an immediate, but also shifts it to put it in the upper half of the destination register.
As LUI does not do exactly what LDI would have done, a new instruction \texttt{LLI} (Load Lower Immediate), takes \texttt{LDI}'s place in the instruction requirements table.

Implementing more instructions than the ones specified in table \vref{table:required-instructions} is optional.
The instruction set implemented by the solution processor is detailed in section \vref{section:instruction-set}.

Once the processor is designed, it must also be verified in a simulation environment as well as in a hardware platform.

\section{Goals}

For this assignment, a set of solution goals were decided upon that should be met in addition the the minimal requirements given in the assignment.
All decisions in the process of completing the assignment were made to further the goals in this section.
The three solution goals are presented here in order of descending importance.

\subsection{Industry Standard Best Practices}

The first goal is that, to the extent that is is practically possible, the solution should conform to the current industry standards in aspects such as implementation style, tooling, best practices and so on.
Using industry standards for the solution processor was a goal for two reasons:
\textit{First}, because the industry has many, many years of combined experience in the field, and using its accumulated knowledge and wisdom for what it is worth can help increase solution stability, ease of development, and avoiding common pitfalls and gotchas, and
\textit{Second}, using well-documented and feature-rich standard libraries facilitate the implementation of many different CPU instructions, which was a prioritized goal for this exercise.

\subsection{A Large and Diverse Instruction Set}

The second solution goal is that the solution processor should support as much of the MIPS I instruction set as possible.
It should ideally behave functionally equivalent as other MIPS I processors.
The reasoning behind this is that the closer the solution processor resembles a regular MIPS processor, the more can existing tools made for other MIPS processors be used for the solution processor.
This allows for things like an existing MIPS-supporting C compiler to compile regular C code to run on the solution processor.

\subsection{Performance}

The third and least prioritized solution goal is that the processor should not be slow.
That means that when an architectural or implementation desicion must be made, the more choice with a more processor performant outcome should be chosen, as long as it does not conflict with the other design goals.
This means that it is ok to sacrifice processor performance if it means that the processor better conforms to industry standards, or that the processor can support more instruction.



ADDING SOME RANDOM NOTES:

when doing number of software accessible registers in the tuts: 5!
