\section{VHDL Test Benches}

In VHDL, one has the opportinity to create test benches to validate VHDL components.
A test bench is a piece of VHDL code that instantiates a component, manipulates its in-signals, and measures the output the component sends out again.
It is the hardware design analog of unit testing in regular software development.
These test benches are typically run in simulator software such as ISim or ModelSim, which simulates hardware in an easily measurable and inspectable environment.

Generally, each component made in VHDL should have a corresponding test bench.
Because a component is typically defined in its own file, a common test bench scheme is to have one file, ``\texttt{my\_entity.vhd}'', which defined the component, and one file, ``\texttt{tb\_my\_entity.vhd}'', which defines the test bench for the component.
Of course, here \texttt{my\_component} is a placeholder name for a component.

In this assignment, each component has an corresponding automatic test bench, which aims to verify correct functionality for a component.
The tests were run in the hardware simulation tool called ISim 12.4 (nt64).
See appendix (TODO: appendix numbers) for detailed results of each test bench simulation.

\subsection{ALU Test Bench}

The alu is continously receiving values for x, y and func, and will use this input to set the correct values for r and the flags.
The test bench include tests for the outputs for each of the alus supported functions.
In addition the subtraction function is tested extra thuraly.
This is because of it's involvement in the branch operations.

\subsection{Branch Controller Test Bench}

The branch controller is a logical unit that will continuously determin if the alu should compare to a zero value, and whether or not to branch.
The test will check each branching operation and test that the compare\_zero and compare_\zero\_value have the correct values.
The test will also check each of these operations with their sensitive flag on and off, and test that branch have the right value.
For good measure we also check that all values is set to 0 when the operation is not a branching operation.
This last test is not really nesseccary in the current processor layout as the branching mux will remove the signal eventually.

\subsection{MUX Test Bench}

The MUX is really quite simple, and so is its test.
The test put two different signals on the input ports and read each of these through each of the imput ports.

\subsection{Pc Test Bench}

The program counter have several inputs which on each rising edge on the clock will reevaluate what value to put out of the program counter.
There are several ways the programcounter can act:
* If reset is set to 1, it will reset pc\_out to 0.
* If both pc enable and reset is set to 1, it should still reset.
* If pc enable is set to 0, it will not load a new value from pc\_in.
* If pc enable is set to 1, pc\_out will be set to the value of pc\_in.
We test that each of these 4 behaviors work as intended.


\subsection{Processor Test Bench}

The processor test bench test that the processor act as expected on a few operations.
First it load some numbers into registers.
When these data are ready, the test do some math on the data.
In the end the data is writen back to memory.
The purpose of this test is not to cover all the operations, as they are already covered in the alu test bench.
The processor test bench will make sure that all of the processors components are working together, and that it is able to execute memory load and store operations.

\subsection{Toplevel Test Bench}

The toplevel test bench defines a short program of 16 binary-coded instructions that it uses to test the processor.
These instructions are helpfully named \texttt{ins0} through \texttt{ins15}.
A comment in the the test bench refers to a certain \texttt{ins.txt}, which supposedly contains a description of the instructions, but it is nowhere to be found.
A manually decoded program listing of this program can be found in listing \vref{listing:toplevel-program}.

\begin{listing}
\begin{center}
\begin{bytefield}[rightcurly=., rightcurlyspace=0pt, leftcurly=., leftcurlyspace=0pt]{32}
\bitheader[endianness=big]{0-16,20,21,25,26,31} \\

\begin{rightwordgroup}{\texttt{lw r0, 1(r0)}}
\begin{leftwordgroup}{\texttt{ins0}}
\bitbox{6}{op: lw \\ \tiny 100011}
& \bitbox{5}{rs: r0 \\ \tiny 00000}
& \bitbox{5}{rt: r1 \\ \tiny 00001}
& \bitbox{16}{immediate: 1 \\ \tiny 0000000000000001}
\end{leftwordgroup}
\end{rightwordgroup} \\

\begin{rightwordgroup}{\texttt{lw r2, 2(r0)}}
\begin{leftwordgroup}{\texttt{ins1}}
\bitbox{6}{op: lw \\ \tiny 100011}
& \bitbox{5}{rs: r0 \\ \tiny 00000}
& \bitbox{5}{rt: r2 \\ \tiny 00010}
& \bitbox{16}{immediate: 1 \\ \tiny 0000000000000010}
\end{leftwordgroup}
\end{rightwordgroup} \\

\begin{rightwordgroup}{\texttt{lw r2, 2(r0)}}
\begin{leftwordgroup}{\texttt{ins2}}
\bitbox{6}{op: lw \\ \tiny 100011}
& \bitbox{5}{rs: r0 \\ \tiny 00000}
& \bitbox{5}{rt: r2 \\ \tiny 00010}
& \bitbox{16}{immediate: 1 \\ \tiny 0000000000000010}
\end{leftwordgroup}
\end{rightwordgroup} \\

\begin{rightwordgroup}{\texttt{add r3, r1, r2}}
\begin{leftwordgroup}{\texttt{ins3}}
\bitbox{6}{op: r\_all \\ \tiny 000000}
& \bitbox{5}{rs: r1 \\ \tiny 00001}
& \bitbox{5}{rt: r2 \\ \tiny 00010}
& \bitbox{5}{rd: r3 \\ \tiny 00011}
& \bitbox{5}{sh: 0 \\ \tiny 00000}
& \bitbox{6}{func: add \\ \tiny 100000}
\end{leftwordgroup}
\end{rightwordgroup} \\

\begin{rightwordgroup}{\texttt{sw r3, 5(r0)}}
\begin{leftwordgroup}{\texttt{ins4}}
\bitbox{6}{op: sw \\ \tiny 101011}
& \bitbox{5}{rs: r0 \\ \tiny 00000}
& \bitbox{5}{rt: r3 \\ \tiny 00011}
& \bitbox{16}{imm: 5 \\ \tiny 0000000000000101}
\end{leftwordgroup}
\end{rightwordgroup} \\

\begin{rightwordgroup}{\texttt{beq r0, r0, 5}}
\begin{leftwordgroup}{\texttt{ins5}}
\bitbox{6}{op: beq \\ \tiny 000100}
& \bitbox{5}{rs: r0 \\ \tiny 00000}
& \bitbox{5}{rt: r0 \\ \tiny 00000}
& \bitbox{16}{imm: 5 \\ \tiny 0000000000000101}
\end{leftwordgroup}
\end{rightwordgroup} \\

\begin{rightwordgroup}{\texttt{sw r3, 3(r0)}}
\begin{leftwordgroup}{\texttt{ins6}}
\bitbox{6}{op: sw \\ \tiny 101011}
& \bitbox{5}{rs: r0 \\ \tiny 00000}
& \bitbox{5}{rt: r3 \\ \tiny 00011}
& \bitbox{16}{imm: 3 \\ \tiny 0000000000000011}
\end{leftwordgroup}
\end{rightwordgroup} \\

\begin{rightwordgroup}{\texttt{sw r3, 4(r0)}}
\begin{leftwordgroup}{\texttt{ins7}}
\bitbox{6}{op: sw \\ \tiny 101011}
& \bitbox{5}{rs: r0 \\ \tiny 00000}
& \bitbox{5}{rt: r3 \\ \tiny 00011}
& \bitbox{16}{imm: 4 \\ \tiny 0000000000000100}
\end{leftwordgroup}
\end{rightwordgroup} \\

\begin{rightwordgroup}{\texttt{sw r3, 6(r0)}}
\begin{leftwordgroup}{\texttt{ins8}}
\bitbox{6}{op: sw \\ \tiny 101011}
& \bitbox{5}{rs: r0 \\ \tiny 00000}
& \bitbox{5}{rt: r3 \\ \tiny 00011}
& \bitbox{16}{imm: 6 \\ \tiny 0000000000000110}
\end{leftwordgroup}
\end{rightwordgroup} \\

\begin{rightwordgroup}{\texttt{sw r3, 7(r0)}}
\begin{leftwordgroup}{\texttt{ins9}}
\bitbox{6}{op: sw \\ \tiny 101011}
& \bitbox{5}{rs: r0 \\ \tiny 00000}
& \bitbox{5}{rt: r3 \\ \tiny 00011}
& \bitbox{16}{imm: 7 \\ \tiny 0000000000000111}
\end{leftwordgroup}
\end{rightwordgroup} \\

\begin{rightwordgroup}{\texttt{lui r3, 6(r0)}}
\begin{leftwordgroup}{\texttt{ins10}}
\bitbox{6}{op: lui \\ \tiny 001111}
& \bitbox{5}{rs: r0 \\ \tiny 00000}
& \bitbox{5}{rt: r3 \\ \tiny 00011}
& \bitbox{16}{imm: 6 \\ \tiny 0000000000000110}
\end{leftwordgroup}
\end{rightwordgroup} \\

\begin{rightwordgroup}{\texttt{sw r3, 8(r0)}}
\begin{leftwordgroup}{\texttt{ins11}}
\bitbox{6}{op: sw \\ \tiny 101011}
& \bitbox{5}{rs: r0 \\ \tiny 00000}
& \bitbox{5}{rt: r3 \\ \tiny 00011}
& \bitbox{16}{imm: 8 \\ \tiny 0000000000001000}
\end{leftwordgroup}
\end{rightwordgroup} \\

\begin{rightwordgroup}{\texttt{add r1, r3, r3}}
\begin{leftwordgroup}{\texttt{ins12}}
\bitbox{6}{op: r\_all \\ \tiny 000000}
& \bitbox{5}{rs: r1 \\ \tiny 00001}
& \bitbox{5}{rt: r3 \\ \tiny 00011}
& \bitbox{5}{rd: r3 \\ \tiny 00011}
& \bitbox{5}{sh: 0 \\ \tiny 00000}
& \bitbox{6}{func: add \\ \tiny 100000}
\end{leftwordgroup}
\end{rightwordgroup} \\

\begin{rightwordgroup}{\texttt{sw r3, 9(r0)}}
\begin{leftwordgroup}{\texttt{ins13}}
\bitbox{6}{op: sw \\ \tiny 101011}
& \bitbox{5}{rs: r0 \\ \tiny 00000}
& \bitbox{5}{rt: r3 \\ \tiny 00011}
& \bitbox{16}{imm: 9 \\ \tiny 0000000000001001}
\end{leftwordgroup}
\end{rightwordgroup} \\

\begin{rightwordgroup}{\texttt{beq r0, r0, -3}}
\begin{leftwordgroup}{\texttt{ins14}}
\bitbox{6}{op: beq \\ \tiny 000100}
& \bitbox{5}{rs: r0 \\ \tiny 00000}
& \bitbox{5}{rt: r0 \\ \tiny 00000}
& \bitbox{16}{imm: -3 \\ \tiny 1111111111111101}
\end{leftwordgroup}
\end{rightwordgroup} \\

\begin{rightwordgroup}{\texttt{sw r3, 10(r0)}}
\begin{leftwordgroup}{\texttt{ins15}}
\bitbox{6}{op: sw \\ \tiny 101011}
& \bitbox{5}{rs: r0 \\ \tiny 00000}
& \bitbox{5}{rt: r3 \\ \tiny 00011}
& \bitbox{16}{imm: 10 \\ \tiny 0000000000001010}
\end{leftwordgroup}
\end{rightwordgroup} \\

\end{bytefield}
\end{center}
\caption{The short program in the toplevel test bench}
\label{listing:toplevel-program}
\end{listing}


\section{Testing on the Spartan-6 FPGA}

It is also possible to run toplevel test benches directly on the FPGA (I think).
TODO: write more about this.

\section{Physical Measurements}

TODO: insert energy efficiency measurements, or something.

This is what the results of such tests may have looked like:
Audible noise test: low audibility. (there are no moving parts)

Energy consumption test:
upper bound: <upper bound of the FPGA>
lower bound: >1 


