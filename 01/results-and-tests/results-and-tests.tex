Presents the results: what has been successfully completed and what did not work.
If any ways around it were found, provide them at this place.
Every solution should be tested for its validity.
This is the place where you will describe what kind of testing you have performed and what the outcome of your tests was.

Testing is an important part of any kind of development, be it software development or other kinds of development.
This section presents the different kinds of tests performed.

\section{VHDL Test Benches}

In VHDL, one has the opportinity to create test benches to validate VHDL components.
A test bench is a piece of VHDL code that instantiates a component, manipulates its in-signals, and measures the output the component sends out again.
It is the hardware design analog of unit testing in regular software development.
These test benches are typically run in simulator software such as ISim or ModelSim, which simulates hardware in an easily measurable and inspectable environment.

Generally, each component made in VHDL should have a corresponding test bench.
Because a component is typically defined in its own file, a common test bench scheme is to have one file, "\texttt{my\_entity.vhd}", which defined the component, and one file, "\texttt{tb\_my\_entity.vhd}", which defines the test bench for the component.
Of course, here \texttt{my\_component} is a placeholder name for a component.

In this assignment, each component has an corresponding automatic test bench, which aims to verify correct functionality for a component.
The tests were run in the hardware simulation tool called ISim (TODO: reference ISim version etc).
See appendix (TODO: appendix numbers) for detailed results of each test bench simulation.

\subsection{ALU Test Bench}

TODO: Something about this test bench.

\subsection{Branch Controller Test Bench}

TODO: Something about this test bench.

\subsection{Processor Test Bench}

TODO: Something about this test bench.

\subsection{Toplevel Test Bench}

The toplevel test bench defines a short program of 16 binary-coded instructions that it uses to test the processor.
These instructions are helpfully named \texttt{ins0} through \texttt{ins15}.
A comment in the the test bench refers to a certain \texttt{ins.txt}, which supposedly contains a description of the instructions, but it is nowhere to be found.
A manually decoded program listing of this program can be found in listing \vref{listing:toplevel-program}.

\begin{listing}
\begin{center}
\begin{bytefield}[rightcurly=., rightcurlyspace=0pt, leftcurly=., leftcurlyspace=0pt]{32}
\bitheader[endianness=big]{0-16,20,21,25,26,31} \\

\begin{rightwordgroup}{\texttt{lw r0, 1(r0)}}
\begin{leftwordgroup}{\texttt{ins0}}
\bitbox{6}{op: lw \\ \tiny 100011}
& \bitbox{5}{rs: r0 \\ \tiny 00000}
& \bitbox{5}{rt: r1 \\ \tiny 00001}
& \bitbox{16}{immediate: 1 \\ \tiny 0000000000000001}
\end{leftwordgroup}
\end{rightwordgroup} \\

\begin{rightwordgroup}{\texttt{lw r2, 2(r0)}}
\begin{leftwordgroup}{\texttt{ins1}}
\bitbox{6}{op: lw \\ \tiny 100011}
& \bitbox{5}{rs: r0 \\ \tiny 00000}
& \bitbox{5}{rt: r2 \\ \tiny 00010}
& \bitbox{16}{immediate: 1 \\ \tiny 0000000000000010}
\end{leftwordgroup}
\end{rightwordgroup} \\

\begin{rightwordgroup}{\texttt{lw r2, 2(r0)}}
\begin{leftwordgroup}{\texttt{ins2}}
\bitbox{6}{op: lw \\ \tiny 100011}
& \bitbox{5}{rs: r0 \\ \tiny 00000}
& \bitbox{5}{rt: r2 \\ \tiny 00010}
& \bitbox{16}{immediate: 1 \\ \tiny 0000000000000010}
\end{leftwordgroup}
\end{rightwordgroup} \\

\begin{rightwordgroup}{\texttt{add r3, r1, r2}}
\begin{leftwordgroup}{\texttt{ins3}}
\bitbox{6}{op: r\_all \\ \tiny 000000}
& \bitbox{5}{rs: r1 \\ \tiny 00001}
& \bitbox{5}{rt: r2 \\ \tiny 00010}
& \bitbox{5}{rd: r3 \\ \tiny 00011}
& \bitbox{5}{sh: 0 \\ \tiny 00000}
& \bitbox{6}{func: add \\ \tiny 100000}
\end{leftwordgroup}
\end{rightwordgroup} \\

\begin{rightwordgroup}{\texttt{sw r3, 5(r0)}}
\begin{leftwordgroup}{\texttt{ins4}}
\bitbox{6}{op: sw \\ \tiny 101011}
& \bitbox{5}{rs: r0 \\ \tiny 00000}
& \bitbox{5}{rt: r3 \\ \tiny 00011}
& \bitbox{16}{imm: 5 \\ \tiny 0000000000000101}
\end{leftwordgroup}
\end{rightwordgroup} \\

\begin{rightwordgroup}{\texttt{beq r0, r0, 5}}
\begin{leftwordgroup}{\texttt{ins5}}
\bitbox{6}{op: beq \\ \tiny 000100}
& \bitbox{5}{rs: r0 \\ \tiny 00000}
& \bitbox{5}{rt: r0 \\ \tiny 00000}
& \bitbox{16}{imm: 5 \\ \tiny 0000000000000101}
\end{leftwordgroup}
\end{rightwordgroup} \\

\begin{rightwordgroup}{\texttt{sw r3, 3(r0)}}
\begin{leftwordgroup}{\texttt{ins6}}
\bitbox{6}{op: sw \\ \tiny 101011}
& \bitbox{5}{rs: r0 \\ \tiny 00000}
& \bitbox{5}{rt: r3 \\ \tiny 00011}
& \bitbox{16}{imm: 3 \\ \tiny 0000000000000011}
\end{leftwordgroup}
\end{rightwordgroup} \\

\begin{rightwordgroup}{\texttt{sw r3, 4(r0)}}
\begin{leftwordgroup}{\texttt{ins7}}
\bitbox{6}{op: sw \\ \tiny 101011}
& \bitbox{5}{rs: r0 \\ \tiny 00000}
& \bitbox{5}{rt: r3 \\ \tiny 00011}
& \bitbox{16}{imm: 4 \\ \tiny 0000000000000100}
\end{leftwordgroup}
\end{rightwordgroup} \\

\begin{rightwordgroup}{\texttt{sw r3, 6(r0)}}
\begin{leftwordgroup}{\texttt{ins8}}
\bitbox{6}{op: sw \\ \tiny 101011}
& \bitbox{5}{rs: r0 \\ \tiny 00000}
& \bitbox{5}{rt: r3 \\ \tiny 00011}
& \bitbox{16}{imm: 6 \\ \tiny 0000000000000110}
\end{leftwordgroup}
\end{rightwordgroup} \\

\begin{rightwordgroup}{\texttt{sw r3, 7(r0)}}
\begin{leftwordgroup}{\texttt{ins9}}
\bitbox{6}{op: sw \\ \tiny 101011}
& \bitbox{5}{rs: r0 \\ \tiny 00000}
& \bitbox{5}{rt: r3 \\ \tiny 00011}
& \bitbox{16}{imm: 7 \\ \tiny 0000000000000111}
\end{leftwordgroup}
\end{rightwordgroup} \\

\begin{rightwordgroup}{\texttt{lui r3, 6(r0)}}
\begin{leftwordgroup}{\texttt{ins10}}
\bitbox{6}{op: lui \\ \tiny 001111}
& \bitbox{5}{rs: r0 \\ \tiny 00000}
& \bitbox{5}{rt: r3 \\ \tiny 00011}
& \bitbox{16}{imm: 6 \\ \tiny 0000000000000110}
\end{leftwordgroup}
\end{rightwordgroup} \\

\begin{rightwordgroup}{\texttt{sw r3, 8(r0)}}
\begin{leftwordgroup}{\texttt{ins11}}
\bitbox{6}{op: sw \\ \tiny 101011}
& \bitbox{5}{rs: r0 \\ \tiny 00000}
& \bitbox{5}{rt: r3 \\ \tiny 00011}
& \bitbox{16}{imm: 8 \\ \tiny 0000000000001000}
\end{leftwordgroup}
\end{rightwordgroup} \\

\begin{rightwordgroup}{\texttt{add r1, r3, r3}}
\begin{leftwordgroup}{\texttt{ins12}}
\bitbox{6}{op: r\_all \\ \tiny 000000}
& \bitbox{5}{rs: r1 \\ \tiny 00001}
& \bitbox{5}{rt: r3 \\ \tiny 00011}
& \bitbox{5}{rd: r3 \\ \tiny 00011}
& \bitbox{5}{sh: 0 \\ \tiny 00000}
& \bitbox{6}{func: add \\ \tiny 100000}
\end{leftwordgroup}
\end{rightwordgroup} \\

\begin{rightwordgroup}{\texttt{sw r3, 9(r0)}}
\begin{leftwordgroup}{\texttt{ins13}}
\bitbox{6}{op: sw \\ \tiny 101011}
& \bitbox{5}{rs: r0 \\ \tiny 00000}
& \bitbox{5}{rt: r3 \\ \tiny 00011}
& \bitbox{16}{imm: 9 \\ \tiny 0000000000001001}
\end{leftwordgroup}
\end{rightwordgroup} \\

\begin{rightwordgroup}{\texttt{beq r0, r0, -3}}
\begin{leftwordgroup}{\texttt{ins14}}
\bitbox{6}{op: beq \\ \tiny 000100}
& \bitbox{5}{rs: r0 \\ \tiny 00000}
& \bitbox{5}{rt: r0 \\ \tiny 00000}
& \bitbox{16}{imm: -3 \\ \tiny 1111111111111101}
\end{leftwordgroup}
\end{rightwordgroup} \\

\begin{rightwordgroup}{\texttt{sw r3, 10(r0)}}
\begin{leftwordgroup}{\texttt{ins15}}
\bitbox{6}{op: sw \\ \tiny 101011}
& \bitbox{5}{rs: r0 \\ \tiny 00000}
& \bitbox{5}{rt: r3 \\ \tiny 00011}
& \bitbox{16}{imm: 10 \\ \tiny 0000000000001010}
\end{leftwordgroup}
\end{rightwordgroup} \\

\end{bytefield}
\end{center}
\caption{The short program in the toplevel test bench}
\label{listing:toplevel-program}
\end{listing}


\section{Testing on the Spartan-6 FPGA}

It is also possible to run toplevel test benches directly on the FPGA (I think).
TODO: write more about this.

\section{Physical Measurements}

TODO: insert energy efficiency measurements, or something.
