Presents the results: what has been successfully completed and what did not work.
If any ways around it were found, provide them at this place.
Every solution should be tested for its validity.
This is the place where you will describe what kind of testing you have performed and what the outcome of your tests was.

Testing is an important part of any kind of development, be it software development or other kinds of development.
This section presents the different kinds of tests performed.

\section{VHDL Test Benches}

In VHDL, one has the opportinity to create test benches to validate VHDL components.
A test bench is a piece of VHDL code that instantiates a component, manipulates its in-signals, and measures the output the component sends out again.
It is the hardware design analog of unit testing in regular software development.
These test benches are typically run in simulator software such as ISim or ModelSim, which simulates hardware in an easily measurable and inspectable environment.

Generally, each component made in VHDL should have a corresponding test bench.
Because a component is typically defined in its own file, a common test bench scheme is to have one file, "\texttt{my\_entity.vhd}", which defined the component, and one file, "\texttt{tb\_my\_entity.vhd}", which defines the test bench for the component.
Of course, here \texttt{my\_component} is a placeholder name for a component.

In this assignment, each component has an corresponding automatic test bench, which aims to verify correct functionality for a component.
The tests were run in the hardware simulation tool called ISim (TODO: reference ISim version etc).
See appendix (TODO: appendix numbers) for detailed results of each test bench simulation.

\section{Testing on the Spartan-6 FPGA}

It is also possible to run toplevel test benches directly on the FPGA (I think).
TODO: write more about this.

\section{Physical Measurements}

TODO: insert energy efficiency measurements, or something.
