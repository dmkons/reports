The assignment was completed and most of the requirements were met.
Three additional design goals were set and prioritized by importance, which heavily influenced the solution.
The solution processor is capable of executing a large number of instructions in a reasonably efficient manner, all the while following the recommended best practices put forth by IEEE and prominent VHDL gurus.
The processor was never successfully run on the FPGA, but due to the extensive testing and careful consideration of VHDL practices, there is every reason to believe that if the processor was properly flashed to the FPGA, it would function as expected.
Of course, this conclusion does not take Moore's law into account.
More realistically, more implementation bugs would have been uncovered.

The assignment was quite difficult, as it introduced a new domain of tools and guidelines that were for the authors quite unfamiliar.
Becoming productive in new and different tool suites takes time, and therefore so did this assignment.
Learning how to write idiomatic VHDL code was probably the hardest part, as it is really only something that can be taught through experience.

An important lesson learned in this assignment is the value of setting realistic goals.
In this assignment, the additional goals set were too ambitious, which led to less focus on the parts of the solution that were required.
It would have been nice, for example, to have been able to test the solution processor on the FPGA, but there was simply not enough time left after all the extra instructions were implemented and properly tested.

In conclusion, it was a fun assignment which was completed to reasonable satisfaction despite less than optimal time management within the group.
Now the twilight of the second assignment has passed and its dawn is upon us, and the lessons learned from its predecessor shall not be forgotten.
