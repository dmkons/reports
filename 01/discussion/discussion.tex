This chapter discusses the performance and test results of the presented solution processor, as well as the assignment it self and the work process around it.

\section{Requirements}

The assignment's main requirement, which was to ``simple multi-cycle MIPS architecture''~\cite[p.114]{compendium}, was met.
It was based on a suggested MIPS-like architecture presented in Figure 4.1~\cite[p.115]{compendium} in the course compendium, as was required.
The required instructions (see table \vref{table:required-instructions} have been implemented for the processor, as well as a vast array of additional and more complex instructions (see figure \vref{table:implemented-instructions}.
The processor was exensively tested at different levels of abstraction using in several simulation environments, which was a requirement that was met.
Finally, the processor was to be tested on a hardware platform.
This requirement was not met.

\section{Additional Goals}

Did we meet the goals? 

\subsection{Industry Standard Best Practices}

Did we meed the goal about conforming to industry standards?
What are the industry standards?

VHDL is a language defined by standards laid forth by IEEE, the Institute of Electrical and Electronics Engineers.
In order to write good VHDL, it is best to use IEEE-sanctioned standard libraries.

\subsubsection{\texttt{numeric\_std}}


For doing arithmetic operations on logic vectors in VHDL, many resources (TODO: cite! websites, tutorials, books) will suggest using the functionality provided in the packages \texttt{ieee.std\_logic\_arith} and \texttt{ieee.std\_logic\_unsigned}.
These packages are non-standard packages defined by a company called Synposis, which makes one of the more popular VHDL tool suites.
Synopsis bundles these packages in their tool distributions, which causes a lot of confusion as to whether or not a part of the standard VHDL library.
Unfortunately, because of this, these non-standard packages have enjoyed wide-spread usage amongst VHDL coders.
This is a problem, because different vendors of VHDL tools have begun supply their own versions of \texttt{std\_logic\_arith} and \texttt{std\_logic\_unsigned}.
Of course, as there is no standard governing these packages, the different versions supplied by different versions have begun to diverge, and are not necessarily compatible with each other.

IEEE has countered this problem by defining new standard packages called \texttt{numeric\_std} and \texttt{numeric\_bit}, which implement similar functionality as the unofficial Synposis packages.
This ensures compatibility across VHDL tools from different vendors.

\texttt{Numeric\_std} has a number of advantages over \texttt{std\_logic\_arith}.
The biggest advantage is that it defines new types for arithmetical logcal vectors, \texttt{signed} and \texttt{unsigned}, as opposed to trying to use \texttt{std\_logic\_vector} directly, which is what the Synopsis packages do.
This is an advantage because combining signed and unsigned arithmetic on the same signals becomes a lot easier.
In fact, the Synposis packages simply assumes all operands are unsigned in all arithmetic operations, which is a bad idea.
A different advantage with the standard packages is that they provide stronger quality assurance through type checking, as they introduce new types.
This is a good thing because it lets developers find bugs earlier, speeding up the development process.

In the solution VHDL code, all arithmetical operations are sanely implemented using \texttt{numeric\_std}.

TODO: citation to this thing might be a good idea: http://vhdlguru.blogspot.no/2010/03/why-library-numericstd-is-preferred.html .

\subsubsection{Test Coverage}

The solution processor has very good simulation test coverage.
Every custom entity in the solution VHDL code has its own extensive test bench, which uses the custom-made convenience functions from the \texttt{test\_utils} package.
Unfortunately, the solution processor was never tested physically on the FPGA, due to unfortunate prioritization in the design goals for the assignment.

To improve the test coverage, the obvious next step is to load the solution processor onto an FPGA and run the same tests on the physical board.

\subsubsection{Additional industry standards?}

TODO: Lets google some stuff and shoehorn it into this report.

\subsection{Instruction Set}

To what extent did we meet the design goal of having a large instruction set?
Was it a good idea to choose this as a design goal? (Hint: no).

\subsection{Performance}

The presented solution processor performs rather well for a sequential simple multi cycle processor.
The clock's maximum speed is (TODO: insert max clock speed).
In testing, it performs reliably in with clock speeds up to X at temperature Y.
Being a multi-cycle processor, most instructions take more than one cycle to complete.
Instructions that only operate on immediates and registers take two cycles to complete.
Instructions that access the data memory require an additional cycle, for a total of three cycles.
This means that under optimal conditions, the processor is capable of X/2 instructions per second (ref IPS).

One way of increasing performance beyond what has been done in the solution is to restructure the logical flow of the architecture to keep the data flow dependencies to a minimum.
Doing so would mean that the processor would need less time to stabilize for each cycle, and as an effect, the clock cycle speed could be increased.
As we all remember from Intel and AMD marketing in the early 2000's, more MHz means more power!

To further increase performance, we could introduce instruction-level parallelism in the form of a pipeline.
This, however, is out of the scope of this exercise.

\section{Energy Efficiency}

Any computer-related report worth its salt should ontain a section that discusses the energy efficiency of the presented solution.
This report is no different.
The solution processor was not designed with energy efficiency as a primary design goal.
It does, however strive for simplicity in its VHDL design, using official IEEE-reccomended industry standards for as many operations as possible.
This lets the synthesizer use its intimate knowledge of the feature set of the target FPGA to create customized hardware configurations that exploit the possibilities the FPGA provides.
This means that dedicated utility slices, block ram sections and similar will be used where possible.
They are typically faster than their custom logic counterparts, and therefore allow for quicker execution.
Although quicker execution by virtue of increased clock speeds does not change the energy efficiency of the processor for finite terminating programs, it may allow other dependant components in a larger system to enter low energy states quicker, and thereby reducing static energy loss in the system as a whole.

To make the processor more energy efficient, sleep states could be introduced.
They allow the processor to enter low energy states when it is not needed, i.e. typically when it is waiting for an external signal in a larger system, and when it has nothing else useful to do.

\section{About the Work Process}

TODO: bla bla, we should have started sooner, yadda yadda, could have done more etc, happy with final outcome.

also we could just go through our logs and see what we actually did and in what order.

THAT'S WHAT LOGS ARE FOR, YEAH.

TODO: attach an appendix of all the raw logs?
