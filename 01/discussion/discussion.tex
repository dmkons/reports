Discuss the assignment and your achievements.
You are free to critically assess your work:
what could have been done better, which way you would choose to go if given the same task again etc.

This chapter discusses the performance and test results of the presented solution processor, as well as the assignment it self and the work process around it.


\section{Energy Efficiency}

Any computer-related report worth its salt should ontain a section that discusses the energy efficiency of the presented solution.
This report is no different.
The solution processor was not designed with energy efficiency as a primary design goal.
It does, however strive for simplicity in its VHDL design, using official IEEE-reccomended industry standards for as many operations as possible.
This lets the synthesizer use its intimate knowledge of the feature set of the target FPGA to create customized hardware configurations that exploit the possibilities the FPGA provides.
This means that dedicated utility slices, block ram sections and similar will be used where possible.
They are typically faster than their custom logic counterparts, and therefore allow for quicker execution.
Although quicker execution by virtue of increased clock speeds does not change the energy efficiency of the processor for finite terminating programs, it may allow other dependant components in a larger system to enter low energy states quicker, and thereby reducing static energy loss in the system as a whole.

To make the processor more energy efficient, sleep states could be introduced.
They allow the processor to enter low energy states when it is not needed, i.e. typically when it is waiting for an external signal in a larger system, and when it has nothing else useful to do.


\section{Performance}

The presented solution processor performs rather well for a sequential simple multi cycle processor.
The clock's maximum speed is (TODO: insert max clock speed).
In testing, it performs reliably in with clock speeds up to X at temperature Y.
Being a multi-cycle processor, most instructions take more than one cycle to complete.
Instructions that only operate on immediates and registers take two cycles to complete.
Instructions that access the data memory require an additional cycle, for a total of three cycles.
This means that under optimal conditions, the processor is capable of X/2 instructions per second (ref IPS).

To further increase performance, we could introduce instruction-level parallelism in the form of a pipeline.
This, however, is out of the scope of this exercise.


\section{About the Work Process}

TODO: bla bla, we should have started sooner, yadda yadda, could have done more etc, happy with final outcome.
